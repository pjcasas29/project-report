\documentclass[10pt,twocolumn,letterpaper]{article}

\usepackage{cvpr}
\usepackage{times}
\usepackage{epsfig}
\usepackage{graphicx}
\usepackage{subfig}
\usepackage{amsmath}
\usepackage{bm}
\usepackage{amssymb}

% Include other packages here, before hyperref.

\graphicspath{{../assets/}}
% If you comment hyperref and then uncomment it, you should delete
% egpaper.aux before re-running latex.  (Or just hit 'q' on the first latex
% run, let it finish, and you should be clear).
\usepackage[breaklinks=true,bookmarks=false]{hyperref}

\cvprfinalcopy % *** Uncomment this line for the final submission

\def\httilde{\mbox{\tt\raisebox{-.5ex}{\symbol{126}}}}

% Pages are numbered in submission mode, and unnumbered in camera-ready
%\ifcvprfinal\pagestyle{empty}\fi
\setcounter{page}{1}
\begin{document}
%%%%%%%%% TITLE
\title{EECS6432: Containerized Orchestration of IoT Lighting Systems}

\author{
  Daniel Parreira\\
  York University\\
  Toronto ON, Canada\\
  \and
  Pedro Casas\\
  York University\\
  Toronto ON, Canada\\
  \and
  Shane Segal\\
  York University\\
  Toronto ON, Canada\\
}

\maketitle
%\thispagestyle{empty}

%%%%%%%%% ABSTRACT
\begin{abstract}
  We have implemented a dockerized, automatically load-balancing IoT system and
  associated web-infrastructure for a networked smart-lighting system. Our
  system can be applied to public spaces with shared resources, potentially
  containing individual actors with competing goals - all while working to
  balance a global setpoint that best satisfies the goals of all independent
  actors in the system.

  Created with horizontal scalability in mind, our software architecture is
  well-suited towards rapid expansion of both users and resources. Our embedded
  IoT devices are connected to our GraphQL API through a Kafka pub-sub system,
  ensuring easy scalability. Our various interaction methods (voice, web app,
  EEG Reader) connect through our Pub-Sub architecture where appropriate, which
  connects them to our orchestrated containerized server architecture. Our PID
  based load-balancer ensures that there are enough server resources available
  and that the requests coming from each available endpoint are satisfied as
  much as possible in the face of possibly competing goals.
\end{abstract}

%%%%%%%%% BODY TEXT
\section{Introduction}
Public spaces with adaptive smart technologies present interesting problems
regarding the satisfiability of concurrent requests for shared resources. We
explore this problem in the context of a smart lighting system in a space such
as a university library or public co-working space. Prior to the introduction of
smart technologies, users might be limited to working in a globally lit
environment, or a locally lit environment that did not effectively consider the
preferences of those around them, or of the energy usage policy of the building
or organization.

Our system enables authenticated and authorized users to make luminance-level
requests about the local lighting level. Our system balances this request with
the requests of spatially co-located individuals who have made prior requests.
These requests can either be trivially compatible with each other (e.g
co-located users making similar requests) or be incompatible (brightness
preferences outside of 10-15\%).

As IoT technology has only recently emerged as a powerful force in the
technology space, there are many situations that are in a suboptimal state. We
attempt to address one such situation through the application architecture
presented here. Our solution works to combine the global needs of a building
such as the overally energy usage or brightness level, as well as local needs of
a user such as the local brightness level being optimal for the situation at
hand. Local optimality for the system at hand can include such things as
brightness for the time of day, color of the light produced. or appropriate
brightness for the task. In addition, we also focus on accessible and innovative
models of user interaction with the goal of reducing the barrier for use of our
system by people of differing physical abilities.

The research challenges we faced include several disparate factors. The load
balancing of our servers through adaptive software is an open problem in
software engineering and one that we apply very modern techniques in software
engineering in order to deal with it. We then have the challenge of balancing
user-local preferences with potentially contradictory preferences of other
co-located users as well as with global preferences that can be set on a
location-wide basis. And finally, we have combined new and experimental human
computer interaction techniques into an integrated technological solution.

\section{Models, Architecture, and Algorithms}
\subsection{Models}
When considering the models in our software system, it is useful to seperate
them into the server software and IoT software. The main model in our IoT
software is that of the User. We require our users to be authenticated with a
verified email and password, in order to create a persistent account for them
that can save preferences and locations, as well as a history of commands in
order to potentially build a model of their behaviour in a future iteration of
this project.

Due to the complex nature of our user interaction methods, we model our commands
as an association between devices, the issuing user, the issuing device type,
and the action that the command will instigate in the lighting system.

\subsection{Architecture}
Our server architecture is composed of a set of containers that each run an
instance of our server software, including GraphQL endpoints and our Kafka
pub-sub architecture.



\end{document}
